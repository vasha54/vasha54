\cvitem{title}{Mechanism for incorporating SVG graphics into the SCADA GALBA HMI.}
\cvitem{tutors}{Eng. Roberto Cardenas Isla,
	Eng. Ariel Guerra Garayta}
\cvitem{description}{ At the University of Informatics Sciences, Faculty 5 has several production centers
	among which is the Center for Industrial Informatics (CEDIN) where the Guardian of the
	ALBA (GALBA), which had the task of providing a solution to the development of a SCADA in conjunction with the
	PDVSA company.
	The Human Machine Interface or HMI (Human Machine Interface) is the module of the GALBA SCADA system
	that allows configuring, supervising and controlling industrial processes. The HMI provides a library of
	graphic components (CG) that make it possible to recreate as realistically as possible the processes that
	they occur in plants. These graphics can be simple geometric primitives or as complex as
	industrial components used in the industry that you want to monitor. The CGs consist of certain
	limitation due to its current product design and implementation that are defined by specialists and leaves
	open to developers its implementation mechanisms and design.
	The work was based on the development of a mechanism for incorporating SVG graphics into the HMI
	of SCADA GALBA with the purpose of solving the limitations present in the graphic components
	of the HMI. The development of the mechanism is supported by the agile XP methodology with the use of the
	Scalable Vector Graphics (SVG) standard and Qt framework.}