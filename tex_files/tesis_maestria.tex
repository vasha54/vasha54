\cvitem{título}{Visualizador de la Interfaz Hombre-Máquina del Sistema ROFLEXIN/LC para Dispositivos Móviles.}
\cvitem{tutores}{Dr.C. Ramón Quiza Sardiñas}
\cvitem{descripción}{  El Centro de Estudios de Fabricación Avanzada y Sostenible (CEFAS) de la Universidad de Matanzas desarrolla el proyecto ROFLEXIN/LC: Sistema robusto, flexible e inteligente, de bajo costo, para monitoreo de sistemas y procesos mecánicos. Este proyecto  esta dirigido fundamentalmente a pequeñas y medianas empresas, donde el uso de las alternativas disponibles en el mercado mundial no es financieramente factible. La presente investigación surge a partir de la necesidad del personal que utiliza el ROFLEXIN/LC, de visualizar la información recopilada por dicho sistema en dispositivos móviles y tablets. El objetivo fundamental es desarrollar una aplicación para móviles capaz de visualizar en tiempo real la información del sistema ROFLEXIN/LC. Para el desarrollo de la solución propuesta se realiza un análisis de las principales herramientas, tecnologías y metodologías que se utilizan en la construcción de un software. El proceso estuvo guiado por el uso de las siguientes herramientas y tecnologías: {\em Visual Paradigm} como herramienta CASE, UML como lenguaje de modelado, JSON como formato de intercambio de datos y Java como lenguaje de programación. Desarrollado con el IDE Eclipse con la extensión ADT. Para validar que los resultados alcanzados fueron los esperados se realizó un conjunto de pruebas. Se logró corregir todas las no conformidades detectadas lo que significa que la solución propuesta está lista para ser desplegada. Finalmente se obtuvo un Visualizador para dispositivos móviles y tablets con sistema operativo Android capaz de representar en tiempo real la información proveniente del sistema ROFLEXIN/LC. }

%\subsection{Vocational}
