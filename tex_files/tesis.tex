\cvitem{título}{Mecanismo de incorporación de gráficos SVG al HMI del SCADA GALBA.}
\cvitem{tutores}{Ing. Roberto Cárdenas Isla,
                 Ing. Ariel Guerra Garayta}
\cvitem{descripción}{  En la Universidad de la Ciencias Informáticas la facultad 5 cuenta con varios centros productivos 
entre los que se encuentra el Centro de Informática Industrial (CEDIN) donde se creó el proyecto Guardián del
ALBA (GALBA), el cual tuvo la tarea de dar solución al desarrollo de un SCADA en conjunto con la
empresa PDVSA.
 La Interfaz Hombre Máquina o HMI (Human Machine Interface) es el módulo del sistema SCADA GALBA
que permite configurar, supervisar y controlar los procesos industriales. El HMI provee una biblioteca de
componentes gráficos (CG) que posibilitan recrear de una manera lo más real posible los procesos que
ocurren en plantas. Estos gráficos pueden ser simples primitivas geométricas o tan complejos como los
componentes industriales utilizados en la industria que se desean supervisar. Los CG constan de cierta
limitación debida a su actual diseño e implementación producto que están definido por especialistas y deja
abierto a los desarrolladores sus mecanismos de implementación y diseño.
 El trabajo se basó en el desarrollo de un mecanismo de incorporación de gráficos SVG al HMI
del SCADA GALBA con el propósito de solucionar las limitaciones presentes en los componentes gráficos
del HMI. El desarrollo del mecanismo está sustentado por la metodología ágil XP con la utilización del
estándar de gráficos vectoriales escalables (SVG) y marco de trabajo Qt.}

%\subsection{Vocational}
