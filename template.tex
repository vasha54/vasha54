%% start of file `template.tex'.
%% Copyright 2006-2015 Xavier Danaux (xdanaux@gmail.com), 2020-2021 moderncv maintainers (github.com/moderncv).
%
% This work may be distributed and/or modified under the
% conditions of the LaTeX Project Public License version 1.3c,
% available at http://www.latex-project.org/lppl/.


\documentclass[11pt,a4paper,sans]{moderncv}        % possible options include font size ('10pt', '11pt' and '12pt'), paper size ('a4paper', 'letterpaper', 'a5paper', 'legalpaper', 'executivepaper' and 'landscape') and font family ('sans' and 'roman')

% moderncv themes
\moderncvstyle{casual}                             % style options are 'casual' (default), 'classic', 'banking', 'oldstyle' and 'fancy'
\moderncvcolor{blue}                               % color options 'black', 'blue' (default), 'burgundy', 'green', 'grey', 'orange', 'purple' and 'red'
%\renewcommand{\familydefault}{\sfdefault}         % to set the default font; use '\sfdefault' for the default sans serif font, '\rmdefault' for the default roman one, or any tex font name
%\nopagenumbers{}                                  % uncomment to suppress automatic page numbering for CVs longer than one page

% character encoding
%\usepackage[utf8]{inputenc}                       % if you are not using xelatex ou lualatex, replace by the encoding you are using
%\usepackage{CJKutf8}                              % if you need to use CJK to typeset your resume in Chinese, Japanese or Korean

% adjust the page margins
\usepackage[scale=0.75]{geometry}
\setlength{\footskip}{122.40004pt}                 % depending on the amount of information in the footer, you need to change this value. comment this line out and set it to the size given in the warning
%\setlength{\hintscolumnwidth}{3cm}                % if you want to change the width of the column with the dates
%\setlength{\makecvheadnamewidth}{10cm}            % for the 'classic' style, if you want to force the width allocated to your name and avoid line breaks. be careful though, the length is normally calculated to avoid any overlap with your personal info; use this at your own typographical risks...

% personal data
\name{Luis Andrés}{Valido Fajardo}
\title{Curriculum}                               % optional, remove / comment the line if not wanted
\born{12 Marzo 1989}                                 % optional, remove / comment the line if not wanted
\address{Antonio Maceo 54}{Código Postal 44430}{Cidra, Unión de Reyes, Matanzas, Cuba}% optional, remove / comment the line if not wanted; the "postcode city" and "country" arguments can be omitted or provided empty
\phone[mobile]{+53 53694742}                   % optional, remove / comment the line if not wanted; the optional "type" of the phone can be "mobile" (default), "fixed" or "fax"
%\phone[fixed]{}
%\phone[fax]{}
\email{luis.valido1989@gmail.com, luis.valido@umcc.cu}                               % optional, remove / comment the line if not wanted
%\homepage{}                         % optional, remove / comment the line if not wanted

% Social icons
%\social[linkedin]{john.doe}                        % optional, remove / comment the line if not wanted
%\social[xing]{john_doe}                           % optional, remove / comment the line if not wanted
%\social[twitter]{jdoe}                             % optional, remove / comment the line if not wanted
\social[github]{vasha54}                              % optional, remove / comment the line if not wanted
%\social[gitlab]{jdoe}                              % optional, remove / comment the line if not wanted
%\social[stackoverflow]{0000000/johndoe}            % optional, remove / comment the line if not wanted
%\social[bitbucket]{jdoe}                           % optional, remove / comment the line if not wanted
%\social[skype]{jdoe}                               % optional, remove / comment the line if not wanted
%\social[orcid]{0000-0000-000-000}                  % optional, remove / comment the line if not wanted
%\social[researchgate]{jdoe}                        % optional, remove / comment the line if not wanted
%\social[researcherid]{jdoe}                        % optional, remove / comment the line if not wanted
\social[telegram]{@vasha54}                            % optional, remove / comment the line if not wanted
\social[whatsapp]{+53 53694742}                     % optional, remove / comment the line if not wanted
%\social[signal]{12345678901}                       % optional, remove / comment the line if not wanted
%\social[matrix]{@johndoe:matrix.org}               % optional, remove / comment the line if not wanted
%\social[googlescholar]{googlescholarid}            % optional, remove / comment the line if not wanted


%\extrainfo{additional information}                 % optional, remove / comment the line if not wanted
\photo[64pt][0.4pt]{picture}                       % optional, remove / comment the line if not wanted; '64pt' is the height the picture must be resized to, 0.4pt is the thickness of the frame around it (put it to 0pt for no frame) and 'picture' is the name of the picture file
%\quote{Some quote}                                 % optional, remove / comment the line if not wanted

% bibliography adjustments (only useful if you make citations in your resume, or print a list of publications using BibTeX)
%   to show numerical labels in the bibliography (default is to show no labels)
%\makeatletter\renewcommand*{\bibliographyitemlabel}{\@biblabel{\arabic{enumiv}}}\makeatother
\renewcommand*{\bibliographyitemlabel}{[\arabic{enumiv}]}
%   to redefine the bibliography heading string ("Publications")
%\renewcommand{\refname}{Articles}

% bibliography with mutiple entries
%\usepackage{multibib}
%\newcites{book,misc}{{Books},{Others}}
%----------------------------------------------------------------------------------
%            content
%----------------------------------------------------------------------------------
\begin{document}
%\begin{CJK*}{UTF8}{gbsn}                          % to typeset your resume in Chinese using CJK
%-----       resume       ---------------------------------------------------------
\makecvtitle

\section{Educación}
\cventry{2008--2013}{Ingeniería en Ciencias Informáticas}{Universidad de las Ciencias de la Informáticas}{La Habana}{}{}  % arguments 3 to 6 can be left empty
\cventry{2014}{Introducción al desarrollo del Sistema Integral de Supervisión y Control Guardían del ALBA}{Universidad de las Ciencias de la Informáticas}{La Habana}{}{}
\cventry{2014}{Introducción al sistema de composición de texto LaTex}{Universidad de las Ciencias de la Informáticas}{La Habana}{}{}
\cventry{2014}{Ciencia Tecnología y Sociedad}{Universidad de las Ciencias de la Informáticas}{La Habana}{}{}
\cventry{2014}{Técnicas de trabajo creativo en grupos}{Universidad de las Ciencias de la Informáticas}{La Habana}{}{Escuela de internacional de verano}
\cventry{2015}{Diplomado de Superación General de los profesionales en adiestramiento}{Universidad de las Ciencias de la Informáticas}{La Habana}{}{}
\cventry{2016--2019}{Maestría en Ingeniería Asistida por Computadora}{Universidad de Matanzas}{Matanzas}{}{}

\section{Educación \emph{Online}}
\cventry{2020}{Curso Html}{Plataforma SoloLearn}{Certificado 1014-20521103}{}{https://www.sololearn.com/Certificate/1014-20521103/pdf/}

\cventry{2020}{Curso C}{Plataforma SoloLearn}{Certificado 1089-20521103}{}{https://www.sololearn.com/Certificate/1089-20521103/pdf/}

\cventry{2020}{Curso CSS}{Plataforma SoloLearn}{Certificado 1023-20521103}{}{https://www.sololearn.com/Certificate/1023-20521103/pdf/}

\cventry{2020}{Curso C++}{Plataforma SoloLearn}{Certificado 1051-20521103}{}{https://www.sololearn.com/Certificate/1051-20521103/pdf/}

\cventry{2020}{Curso SQL}{Plataforma SoloLearn}{Certificado 1060-20521103}{}{https://www.sololearn.com/Certificate/1060-20521103/pdf/}

\cventry{2020}{Curso Java}{Plataforma SoloLearn}{Certificado 1068-20521103}{}{https://www.sololearn.com/Certificate/1068-20521103/pdf/}

\cventry{2020}{Curso Ruby}{Plataforma SoloLearn}{Certificado 1081-20521103}{}{https://www.sololearn.com/Certificate/1081-20521103/pdf/}

\cventry{2020}{Curso PHP}{Plataforma SoloLearn}{Certificado 1059-20521103}{}{https://www.sololearn.com/Certificate/1059-20521103/pdf/}

\cventry{2020}{Curso JQuery}{Plataforma SoloLearn}{Certificado 1082-20521103}{}{https://www.sololearn.com/Certificate/1082-20521103/pdf/}

\cventry{2020}{Curso JavaScript}{Plataforma SoloLearn}{Certificado 1024-20521103}{}{https://www.sololearn.com/Certificate/1024-20521103/pdf/}

\cventry{2020}{Curso C\#}{Plataforma SoloLearn}{Certificado 1080-20521103}{}{https://www.sololearn.com/Certificate/1080-20521103/pdf/}

\cventry{2020}{Curso Swift 4}{Plataforma SoloLearn}{Certificado 1075-20521103}{}{https://www.sololearn.com/Certificate/1075-20521103/pdf/}

\cventry{2020}{Curso Python 3}{Plataforma SoloLearn}{Certificado 1073-20521103}{}{https://www.sololearn.com/Certificate/1073-20521103/pdf/}

\cventry{2020}{Curso React + Redux}{Plataforma SoloLearn}{Certificado 1097-20521103}{}{https://www.sololearn.com/Certificate/1097-20521103/pdf/}

\cventry{2020}{Curso Angular + NestJS}{Plataforma SoloLearn}{Certificado 1092-20521103}{}{https://www.sololearn.com/Certificate/1092-20521103/pdf/}

\cventry{2020}{Curso \emph{Data Science with Python}}{Plataforma SoloLearn}{Certificado 1093-20521103}{}{https://www.sololearn.com/Certificate/1093-20521103/pdf/}

\cventry{2021}{Curso \emph{Machine Learning}}{Plataforma SoloLearn}{Certificado 1094-20521103}{}{https://www.sololearn.com/Certificate/1094-20521103/pdf/}

\cventry{2021}{Curso \emph{Python for Beginners}}{Plataforma SoloLearn}{Certificado 1157-20521103}{}{https://www.sololearn.com/Certificate/1157-20521103/pdf/}

\cventry{2021}{Curso \emph{Intermediate Python}}{Plataforma SoloLearn}{Certificado 1158-20521103}{}{https://www.sololearn.com/Certificate/1158-20521103/pdf/}

\cventry{2021}{Curso \emph{Python Data Structures}}{Plataforma SoloLearn}{Certificado 1159-20521103}{}{https://www.sololearn.com/Certificate/1159-20521103/pdf/}

\cventry{2021}{Curso \emph{Responsive Web Design}}{Plataforma SoloLearn}{Certificado 1159-20521103}{}{https://www.sololearn.com/Certificate/1162-20521103/pdf/}

\cventry{2021}{Curso \emph{Kotlin}}{Plataforma SoloLearn}{Certificado 1160-20521103}{}{https://www.sololearn.com/Certificate/1160-20521103/pdf/}

\cventry{2021}{Curso \emph{Go}}{Plataforma SoloLearn}{Certificado 1164-20521103}{}{https://www.sololearn.com/certificates/course/en/20521103/1164/landscape/png}

\cventry{2021}{Curso \emph{Python for Data Science}}{Plataforma SoloLearn}{Certificado 1161-20521103}{}{https://www.sololearn.com/certificates/course/en/20521103/1161/landscape/png}

\cventry{2021}{Curso \emph{Coding for Marketers}}{Plataforma SoloLearn}{Certificado 1165-20521103}{}{https://www.sololearn.com/Certificate/1165-20521103/pdf/}

\cventry{2021}{Curso R}{Plataforma SoloLearn}{Certificado 1147-20521103}{}{https://www.sololearn.com/certificates/course/en/20521103/1147/landscape/png}

\cventry{2021}{El modelo de Educación a Distancia de la Educación Superior Cubana en el contexto de la Universidad de Matanzas}{Universidad de Matanzas}{Matanzas}{Curso Postgrado virtual, 1 credito }{http://evead.umcc.cu/course/view.php?id=105}

\cventry{2021}{Curso \emph{Python for Finance}}{Plataforma SoloLearn}{Certificado 1139-20521103}{}{https://www.sololearn.com/Certificate/1139-20521103/pdf/}

\cventry{2021}{Curso \emph{Game Development with JS}}{Plataforma SoloLearn}{Certificado 1140-20521103}{}{https://www.sololearn.com/Certificate/1140-20521103/pdf/}

\cventry{2022}{Curso \emph{Web Development Fundamentals course}}{Plataforma SoloLearn}{Certificado 1141-20521103}{}{https://www.sololearn.com/Certificate/1141-20521103/pdf/}
\section{Tesis}
\cvitem{título}{Mecanismo de incorporación de gráficos SVG al HMI del SCADA GALBA.}
\cvitem{tutores}{Ing. Roberto Cárdenas Isla,
                 Ing. Ariel Guerra Garayta}
\cvitem{descripción}{  En la Universidad de la Ciencias Informáticas la facultad 5 cuenta con varios centros productivos 
entre los que se encuentra el Centro de Informática Industrial (CEDIN) donde se creó el proyecto Guardián del
ALBA (GALBA), el cual tuvo la tarea de dar solución al desarrollo de un SCADA en conjunto con la
empresa PDVSA.
 La Interfaz Hombre Máquina o HMI (Human Machine Interface) es el módulo del sistema SCADA GALBA
que permite configurar, supervisar y controlar los procesos industriales. El HMI provee una biblioteca de
componentes gráficos (CG) que posibilitan recrear de una manera lo más real posible los procesos que
ocurren en plantas. Estos gráficos pueden ser simples primitivas geométricas o tan complejos como los
componentes industriales utilizados en la industria que se desean supervisar. Los CG constan de cierta
limitación debida a su actual diseño e implementación producto que están definido por especialistas y deja
abierto a los desarrolladores sus mecanismos de implementación y diseño.
 El trabajo se basó en el desarrollo de un mecanismo de incorporación de gráficos SVG al HMI
del SCADA GALBA con el propósito de solucionar las limitaciones presentes en los componentes gráficos
del HMI. El desarrollo del mecanismo está sustentado por la metodología ágil XP con la utilización del
estándar de gráficos vectoriales escalables (SVG) y marco de trabajo Qt.}

%\subsection{Vocational}


\section{Tesis de maestría}
\cvitem{título}{Visualizador de la Interfaz Hombre-Máquina del Sistema ROFLEXIN/LC para Dispositivos Móviles.}
\cvitem{tutores}{Dr.C. Ramón Quiza Sardiñas}
\cvitem{descripción}{  El Centro de Estudios de Fabricación Avanzada y Sostenible (CEFAS) de la Universidad de Matanzas desarrolla el proyecto ROFLEXIN/LC: Sistema robusto, flexible e inteligente, de bajo costo, para monitoreo de sistemas y procesos mecánicos. Este proyecto  esta dirigido fundamentalmente a pequeñas y medianas empresas, donde el uso de las alternativas disponibles en el mercado mundial no es financieramente factible. La presente investigación surge a partir de la necesidad del personal que utiliza el ROFLEXIN/LC, de visualizar la información recopilada por dicho sistema en dispositivos móviles y tablets. El objetivo fundamental es desarrollar una aplicación para móviles capaz de visualizar en tiempo real la información del sistema ROFLEXIN/LC. Para el desarrollo de la solución propuesta se realiza un análisis de las principales herramientas, tecnologías y metodologías que se utilizan en la construcción de un software. El proceso estuvo guiado por el uso de las siguientes herramientas y tecnologías: {\em Visual Paradigm} como herramienta CASE, UML como lenguaje de modelado, JSON como formato de intercambio de datos y Java como lenguaje de programación. Desarrollado con el IDE Eclipse con la extensión ADT. Para validar que los resultados alcanzados fueron los esperados se realizó un conjunto de pruebas. Se logró corregir todas las no conformidades detectadas lo que significa que la solución propuesta está lista para ser desplegada. Finalmente se obtuvo un Visualizador para dispositivos móviles y tablets con sistema operativo Android capaz de representar en tiempo real la información proveniente del sistema ROFLEXIN/LC. }

%\subsection{Vocational}

\section{Experiencia}
\cventry{2009--2015}{Sistema Integral de Supervisión y Control Guardían del ALBA (SCADA GALBA)}{Desarrollador}{La Habana}{Universidad 
de las Ciencias Informáticas, Centro de Informática Industrial}{Desarrollador del módulo interfaz hombre-máquina (HMI) perteneciente 
al proyecto SCADA GALBA.  }
\cventry{2014--2016}{Sistema Automatizado Industrial basado en GNU/LINUX (SCADA SAINUX)}{Desarrollador}{La Habana}{Universidad 
de las Ciencias Informáticas, Centro de Informática Industrial}{Desarrollador del módulo interfaz hombre-máquina (HMI) perteneciente 
al proyecto SCADA SAINUX.}
\cventry{2016--2023}{Departamento de Recursos Para el Aprendizaje}{Desarrollador}{Matanzas}{Universidad 
	de Matanzas}{Desarrollador de extensiones a los Entornos Virtuales de Aprendizajes de la Universidad de Matanzas, de aplicaciones educativas para dispositivos con sistema operativo Android. Administrador de los Entornos Virtuales de Aprendizajes de la Universidad de Matanzas. Instalación y despliegue de sitios web con la utilización de CMS y LMS}
\cventry{2016--2022}{Departamento de Informática}{Profesor adjunto}{Matanzas}{Universidad 
	de Matanzas}{Categoría docente: Instructor}
\cventry{2021--2022}{COMBIOMED Tecnología Médica Digital}{Desarrollador}{La Habana}{}{Desarrollador del módulo interfaz hombre-máquina (HMI) del software para el control y monitoreo del ventilador artificial para los pacientes de COVID-19.}
\cventry{2021--2025}{Instituto Preuniversitario de Ciencias Exactas de Matanzas}{Entrenador de Concurso de Informática}{Matanzas}{}{}
\cventry{2023--2024}{Departamento de Redes}{Especialista B}{Matanzas}{Universidad 
	de Matanzas}{Soporte técnico de los Entornos Virtuales de Aprendizajes de la Universidad de Matanzas. Instalación y despliegue de sitios web con la utilización de LMS Moodle}
\cventry{2023--2024}{INNOVAT}{Desarrollador}{Matanzas}{}{Diseño e implementación de la aplicación PSGAME+ para dispositivos móviles con Android}
\cventry{2023--2025}{Merchise Autrement}{Desarrollador}{}{\href{https://www.merchise.org/es}{Soluciones Empresariales sobre odoo, consultoría e implantación}}{Implementación de funcionalidades. Correcciones de no conformidades}
\cventry{2024--2025}{Desoft}{Desarrollador}{}{Soluciones Empresariales sobre odoo}{Implementación de módulos.}
\cventry{2025--2025}{Digital Learning}{Desarrollador}{}{\href{https://github.com/DigitalLearning}{Digital Learning}}{Implementación de módulo para \href{https://github.com/DigitalLearning/MDL_tool_quickcoursecreation}{Creación rápida de cursos para plataforma Moodle y Totara}.}






%\subsection{Miscellaneous}
%\cventry{year--year}{Job title}{Employer}{City}{}{General description no longer than 1--2 lines}

\section{Docencia impartida}


\cventry{2016--2016}{Lenguaje de Programación Python}{Ingeniería  Informática}{Matanzas}{Universidad 
	de Matanzas}{5to año}
\cventry{2016--2022}{Arquitectura de Computadoras}{Ingeniería  Informática}{Matanzas}{Universidad 
	de Matanzas}{2do año}
\cventry{2017--2020}{Preparación Especifica para Concursos ICPC}{Ingeniería  Informática}{Matanzas}{Universidad 
	de Matanzas}{1er, 2do y 3er  año}
\cventry{2019--2019}{Curso de Android}{Licenciatura en Educación Informática}{Matanzas}{Universidad 
	de Matanzas}{3ro año}
\cventry{2019--2022}{Introducción a la programación}{Ingeniería  Informática}{Matanzas}{Universidad 
	de Matanzas}{1er año}
\cventry{2020--2022}{Diseño y Programación Orientada a Objetos}{Ingeniería  Informática}{Matanzas}{Universidad 
	de Matanzas}{1er año}
\section{Tutoría de tesis}
\cventry{2015}{Sistema para la gestión de la información de los expedientes técnicos en el Centro de  Informática Industrial}{Yadriel Cuesta Hechavarría}{Universidad de las Ciencias Informáticas}{La Habana}{Trabajo de diploma para optar por el título de Ingeniero en Ciencias Informáticas}
\cventry{2015}{Procesador de estado de red para adecuaciones del SCADA-GALBA al sector eléctrico.}{Carlos Manuel Castillo Chacón, Jose Manuel Acevedo Medina}{Universidad de las Ciencias Informáticas}{La Habana}{Trabajo de diploma para optar por el título de Ingeniero en Ciencias Informáticas}
\cventry{2016}{Componentes gráficos para la representación de accesorios de tuberías en el SCADA SAINUX}{Rosmery Pedraza Ceballo}{Universidad de las Ciencias Informáticas}{La Habana}{Trabajo de diploma para optar por el título de Ingeniero en Ciencias Informáticas}
\cventry{2016}{Componentes gráficos para la representación de válvulas y bombas en el SCADA SAINUX}{Claudia María González Fernández}{Universidad de las Ciencias Informáticas}{La Habana}{Trabajo de diploma para optar por el título de Ingeniero en Ciencias Informáticas}
\cventry{2016}{Cliente escritorio para el entorno de visualización de la interfaz hombre máquina  del SCADA SAINUX}{Alik Ramón del Risco del Risco}{Universidad de las Ciencias Informáticas}{La Habana}{Trabajo de diploma para optar por el título de Ingeniero en Ciencias Informáticas}

\cventry{2016}{Componentes gráficos para la representación de interruptores y selectores en el SCADA SAINUX}{Javier Bravo Calzado}{Universidad de las Ciencias Informáticas}{La Habana}{Trabajo de diploma para optar por el título de Ingeniero en Ciencias Informáticas}


\cventry{2016}{Desarrollo de componentes gráficos de almacenamiento en el SCADA SAINUX }{Luis Orlando Cejas}{Universidad de las Ciencias Informáticas}{La Habana}{Trabajo de diploma para optar por el título de Ingeniero en Ciencias Informáticas}

\cventry{2016}{Mecanismo de serialización para la información gestionada en el editor SCADA SAINUX }{Alvaro Denis Acosta Quesada}{Universidad de las Ciencias Informáticas}{La Habana}{Trabajo de diploma para optar por el título de Ingeniero en Ciencias Informáticas}

\cventry{2022}{Plugin para medir los indicadores de calidad de los cursos de Moodle}{Hamsel Brea García}{Universidad Matanzas}{Matanzas}{Trabajo de diploma para optar por el título de Ingeniero Informático}

\cventry{2022}{Sistema informático para la gestión de recursos humanos y servicios del Proyecto de Creación Artística Vigía}{Ramsey Ricardo Busto Martínez}{Universidad Matanzas}{Matanzas}{Trabajo de diploma para optar por el título de Ingeniero Informático}

\cventry{2023}{Sistema de pruebas antencionales para la termoeléctrica “Antonio Guiteras”.}{Ronald Guillén Sintes}{Universidad Matanzas}{Matanzas}{Trabajo de diploma para optar por el título de Ingeniero Informático}
\section{Oponencia de tesis}
\cventry{2017}{Software para la gestión de la información del cliente interno en sistemas hoteleros desde la perspectiva del endonarketing}{Fabianne Díaz Santiago}{Universidad de Matanzas}{Matanzas}{Trabajo de diploma para optar por el título de Ingeniería Informática}
\cventry{2019}{Sistema de Información informático para la Dirección de la Universidad de Matanzas, SID}{Rasiel Valdespino Jiménez}{Universidad de Matanzas}{Matanzas}{Trabajo de diploma para optar por el título de Ingeniería Informática}

\section{Lenguajes}
\cvitemwithcomment{Español}{Nativo}{}
\cvitemwithcomment{Inglés}{Competencia profesional}{}
\section{Herramientas y tecnologías}
\cvitemwithcomment{Lenguaje de programación}{C, C++, Java, JavaScript, Html5, CSS, Python, PHP, SQL, Bash}{}
\cvitemwithcomment{Sistema Operativo}{Debian, Ubuntu, Window, Android}{}
\cvitemwithcomment{Marco de trabajo}{Qt, Yii, CodeIngiter, JQuery, AngularJS, W3CSS, Bootstrap, Django}{}

\cvitemwithcomment{Sistema de gestión de contenidos}{Joomla, Moodle, WordPress}{}

\cvitemwithcomment{Gestor de base datos}{Postgres, MySql}{}

\cvitemwithcomment{Diseño gráfico 2D}{Gimp, Inkscape}{}

\cvitemwithcomment{Herramienta CASE}{Visual Paradigm}{}

\cvitemwithcomment{Servidor web}{Apache, XAMPP}{}

\cvitemwithcomment{Procesador de texto}{LaTex}{}

\cvitemwithcomment{Control de versiones}{Git}{}
\section{Publicaciones}
\cventry{2017}{Mecanismo de automatización de las copias de seguridad del Entorno Virtual de Aprendizaje de la Universidad de Matanzas}{VIII Convención Científica Internacional  de la Universidad de Matanzas }{V Taller Internacional de Ingenierías }{Matanzas}{Publicado en citc.umcc.cu con el código ISBN: 978-959-16-3296-8}
\cventry{2017}{"Cesar y los números": recurso informático para favorecer el aprendizaje de la operación suma, en el primer grado}{XIX Evento Internacional "La Matemática, la Estadística y la Computación"}{MATECOMPU 2017}{Varadero}{Publicado en citc.umcc.cu con el código ISBN: 978-959-16-3671-3}
\cventry{2018}{{\em Herramienta de apoyo a la asignatura probabilidad y estadísticas para graficar distribuciones de probabilidad en la modalidad de estudio educación a distancia de la Universidad de Matanzas}}{Revista Opuntia Brava, Grupo I Clasificación cubana}{}{Las Tunas}{Publicado en el libro Ciencia e innovación tecnológica vol II, en el capítulo {\em Educación a distancia}. Con ISBN para la obra completa 978-959-7225-26-3, ISBN para el capítulo 978-959-7225-34-8}
\cventry{2018}{Desarrollo de los componentes gráficos curva cuadratica y cúbica de Bezier}{Monografías}{Universidad de Matanzas}{Matanzas}{Publicado en monografia.umcc.cu (http://monografias.umcc.cu/monos/2018/FCT/mo1871.pdf)}
\cventry{2018}{Interfaz gráfica de usuario para la interacción con los módulos del SCADA SAINUX (VSERSAI)}{Monografías}{Universidad de Matanzas}{Matanzas}{Publicado en monografia.umcc.cu (http://monografias.umcc.cu/monos/2018/FCT/mo1872.pdf)}
\cventry{2018}{Visualizador de la Interfaz Hombre-Máquina del sistema ROFLEXIN/LC para dispositivos móviles con Android}{Monografías}{Universidad de Matanzas}{Matanzas}{Publicado en monografia.umcc.cu (http://monografias.umcc.cu/monos/2018/FCT/mo1876.pdf)}
\cventry{2019}{Arquitectura de monitoreo inteligente y de bajo costo para sistema y procesos mecánimos}{IX Conferencia Científica Internacional}{Universidad de Holguín}{Holguín}{ISBN:978-959-7237-34-1}
\cventry{2020}{MiGimusic: APK para favorecer la educación musical en la Gimnasia Rítmica pioneril}{Congreso Internacional online CUBAMTRICIDAD}{Universidad de  Ciencias de la Cultura Física y el Deporte \emph{Manuel Fajardo}}{La Habana}{ISBN:978-959-07-2395-7}
\cventry{2020}{La innovación tecnológica para la educación musical en la Gimnasia Rítmica Musical}{1er Taller Nacional de Innovación ONLINE}{Instituto Nacional de Deportes Educación Física y Recreación}{La Habana}{ISBN:978-959-203-237-8}

\cventry{2021}{El uso de dispositivos móviles en la docencia universitaria cubana}{Academia.edu}{https://www.academia.edu/47857643/EL\_USO\_DE\_DISPOSITIVOS\_MOVILES}{}{}

\cventry{2021}{El uso de dispositivos móviles. Una experiencia innovadora en la formación del profesional}{\emph{Libro} La ciencia de la Universidad de Matanzas en el enfrentamiento a la COVID-19 }{Editorial UM}{}{}

\section{Eventos}
\cventry{2008}{Copa Pascal}{Universidad de las Ciencias de la Informáticas}{La Habana}{}{Concursante}
\cventry{2009}{Copa Pascal}{Universidad de las Ciencias de la Informáticas}{La Habana}{}{Concursante}
\cventry{2010}{Copa Pascal}{Universidad de las Ciencias de la Informáticas}{La Habana}{}{Concursante}
\cventry{2011}{Copa Pascal}{Universidad de las Ciencias de la Informáticas}{La Habana}{}{Concursante}
\cventry{2011}{Final Cubana de ACM-ICPC}{Universidad de las Ciencias de la Informáticas}{La Habana}{}{Concursante}
\cventry{2011}{Mi Web por Cuba}{Universidad de las Ciencias de la Informáticas}{La Habana}{}{Relevante}
\cventry{2012}{Copa Pascal}{Universidad de las Ciencias de la Informáticas}{La Habana}{}{Concursante}
\cventry{2012}{Jornada Científica Estudiantil}{Universidad de las Ciencias de la Informáticas}{La Habana}{}{Tres relevantes, un destacado y tres menciones}
\cventry{2012}{Mi Web por Cuba}{Universidad de las Ciencias de la Informáticas}{La Habana}{}{Destacado y mención}
\cventry{2013}{Jornada Científica Estudiantil}{Universidad de las Ciencias de la Informáticas}{La Habana}{}{Destacado}
\cventry{2015}{Pachamama Game Jam}{Universidad de las Ciencias de la Informáticas}{La Habana}{}{Tercer lugar}
\cventry{2017}{VI Copa de Programación UM}{Universidad de Matanzas}{Matanzas}{}{Tercer lugar, Entrenador}
\cventry{2017}{Jornada Cientifica CEPROMEDE}{Centro Provincial de Medicina del Deporte}{Matanzas}{}{}
\cventry{2017}{Final Cubana de ACM-ICPC}{Universidad Técnologica de la Habana José Antonio Echeverría}{Habana}{}{Entrenador}
\cventry{2017}{XIX Evento Internacional "La enseñanza de la Matemática, la Estadística y la Computación"}{Universidad de Matanzas}{Matanzas}{Ponencia: {\em César y los números}: recurso informático para favorecer el aprendizaje de la operación suma, en el primer grado}{}
\cventry{2017}{Seminario Sistemas Ciberfísicos: Modelación y Control}{Universidad de Matanzas}{Matanzas}{}{}
\cventry{2018}{VII Copa de Programación UM}{Universidad de Matanzas}{Matanzas}{}{Entrenador}
\cventry{2018}{XIV Conferencia Científica Metodológica de la Universidad de Matanzas}{Universidad de Matanzas}{Matanzas}{}{}
\cventry{2018}{Concurso Local ACM-ICPC}{Universidad de Matanzas}{Matanzas}{2do Lugar}{Entrenador}
\cventry{2018}{Concurso Nacional ACM-ICPC}{Instituto Técnico Militar "Jose Martí"}{La Habana}{}{Entrenador}
\cventry{2018}{Fórum Universitario de Ciencias y Técnicas}{Universidad de Matanzas}{Matanzas}{Ponenecia:{\em Aplicación para el monitoreo de sistemas y procesos mecánicos desde dispositivos móviles}}{Mención}
\cventry{2018}{II Simposio Internacional de la Red de Investigadores de la Ciencia y la Técnica:{\em Ciencia e Innovación Tecnológica}}{}{Las Tunas}{Ponencia:{\em Herramienta de apoyo a la asignatura probabilidad y estadísticas para graficar distribuciones de probabilidad en la modalidad de estudio educación a distancia de la Universidad de Matanzas}}{}
\cventry{2019}{Seminario {\em Computational intelligence for modeling and control of industrial cyber physical system: Research challenges and opportunities.}}{Universidad de Matanzas}{Matanzas}{}{}
\cventry{2019}{VIII Copa de Programación UM}{Universidad de Matanzas}{Matanzas}{}{Entrenador y Organizador}
\cventry{2019}{IX Convención Científica Internacional}{VI Taller Internacional de Ingenierías}{Universidad de Matanzas}{Matanzas}{Visualizador de la interfaz hombre-máquina del sistema ROFLEXIN/LC para dispositivos móviles}
\cventry{2019}{IX Conferencia Científica Internacional}{}{Universidad de Holguín}{Holguín}{Arquitectura de monitoreo inteligente y de bajo costo para sistema y procesos mecánimos}
\cventry{2019}{Concurso Nacional ICPC}{Universidad de Matanzas}{Matanzas}{}{Entrenador}
\cventry{2019}{Final Regional del Caribe ICPC}{Universidad de Oriente}{Santiago de Cuba}{}{Entrenador}
\cventry{2020}{Primer Taller Online de Innovación}{La innovación tecnológica para la educación musical en la Gimnasia Rítmica pioneril}{INDER}{}{}
\cventry{2021}{ICPC Caribbean Finals (Qualifier)}{Dirección Provincial de Educación}{Matanzas}{Equipos Preuniversitario - Bronce}{Entrenador}
\cventry{2022}{ICPC Caribbean Finals (Qualifier)}{Universidad de Matanzas}{Matanzas}{}{Entrenador}
\cventry{2022}{Final Regional del Caribe ICPC}{Universidad de las Ciencias Informáticas}{La Habana}{}{Entrenador}
\cventry{2023}{ICPC Caribbean Finals (Qualifier)}{Universidad de Matanzas}{Matanzas}{Equipos Preuniversitario - Bronce}{Entrenador}
\cventry{2023}{Final Regional del Caribe ICPC}{Universidad de las Ciencias Informáticas}{La Habana}{}{Entrenador}
\section{Investigaciones}
\cventry{2020}{Educar con José Martí en la escuela cubana}{Facultad de Educación. Universidad de Matanzas}
{}{}{Colaborador}
\section{Reconocimientos}
\cventry{2013}{Resultados productivos}{}{La Habana}{Universidad 
de las Ciencias Informáticas, Facultad 5}{}
\cventry{2014}{Desarrollador de Software}{}{La Habana}{Universidad 
de las Ciencias Informáticas, Centro de Informática Industrial}{}
\cventry{2014}{Destacado}{}{La Habana}{Universidad 
de las Ciencias Informáticas, Centro de Informática Industrial}{}
\cventry{2017}{Invitado}{Final Regional del Caribe ACM-ICPC}{Villa Clara}{Universidad Central de las Villas "Marta Abreu"}{}
\cventry{2018}{Invitado}{Final Regional del Caribe ACM-ICPC}{La Habana}{Universidad Ciencias Informática}{}
\cventry{2019}{Reconocimiento}{Trabajador destacado por la calidad}{Universidad de Matanzas}{Matanzas}{}

\pagebreak

% Curriculum en ingles

\makecvtitle

\section{Education}
\cventry{2008--2013}{Computer Science Engineering}{University of Computer Science}{Havana}{}{} % arguments 3 to 6 can be left empty
\cventry{2014}{Introduction to the development of the ALBA Comprehensive Supervision and Control System}{University of Informatics Sciences}{Havana}{}{}
\cventry{2014}{Introduction to the LaTex text composition system}{University of Informatics Sciences}{Havana}{}{}
\cventry{2014}{Science Technology and Society}{University of Informatics Sciences}{Havana}{}{}
\cventry{2014}{Creative group work techniques}{University of Informatics Sciences}{Havana}{}{International Summer School}
\cventry{2015}{General Improvement Course for professionals in training}{University of Informatics Sciences}{Havana}{}{}
\cventry{2016--2019}{ Master in Computer Aided Engineering}{ University of Matanzas }{Matanzas}{}{}
\section{Education Online}
\cventry{2020}{Course Html}{Sololearn platform}{Certificate 1014-20521103}{}{https://www.sololearn.com/Certificate/1014-20521103/pdf/}

\cventry{2020}{Course C}{Sololearn platform}{Certificate 1089-20521103}{}{https://www.sololearn.com/Certificate/1089-20521103/pdf/}

\cventry{2020}{Course CSS}{Sololearn platform}{Certificate 1023-20521103}{}{https://www.sololearn.com/Certificate/1023-20521103/pdf/}

\cventry{2020}{Course C++}{Sololearn platform}{Certificate 1051-20521103}{}{https://www.sololearn.com/Certificate/1051-20521103/pdf/}

\cventry{2020}{Course SQL}{Sololearn platform}{Certificate 1060-20521103}{}{https://www.sololearn.com/Certificate/1060-20521103/pdf/}

\cventry{2020}{Course Java}{Sololearn platform}{Certificate 1068-20521103}{}{https://www.sololearn.com/Certificate/1068-20521103/pdf/}

\cventry{2020}{Course Ruby}{Sololearn platform}{Certificate 1081-20521103}{}{https://www.sololearn.com/Certificate/1081-20521103/pdf/}

\cventry{2020}{Course PHP}{Sololearn platform}{Certificate 1059-20521103}{}{https://www.sololearn.com/Certificate/1059-20521103/pdf/}

\cventry{2020}{Course JQuery}{Sololearn platform}{Certificate 1082-20521103}{}{https://www.sololearn.com/Certificate/1082-20521103/pdf/}

\cventry{2020}{Course JavaScript}{Sololearn platform}{Certificate 1024-20521103}{}{https://www.sololearn.com/Certificate/1024-20521103/pdf/}

\cventry{2020}{Course C\#}{Sololearn platform}{Certificate 1080-20521103}{}{https://www.sololearn.com/Certificate/1080-20521103/pdf/}

\cventry{2020}{Course Swift 4}{Sololearn platform}{Certificate 1075-20521103}{}{https://www.sololearn.com/Certificate/1075-20521103/pdf/}

\cventry{2020}{Course Python 3}{Sololearn platform}{Certificate 1073-20521103}{}{https://www.sololearn.com/Certificate/1073-20521103/pdf/}

\cventry{2020}{Course React + Redux}{Sololearn platform}{Certificate 1097-20521103}{}{https://www.sololearn.com/Certificate/1097-20521103/pdf/}

\cventry{2020}{Course Angular + NestJS}{Sololearn platform}{Certificate 1092-20521103}{}{https://www.sololearn.com/Certificate/1092-20521103/pdf/}

\cventry{2020}{Course Data Science with Python}{Sololearn platform}{Certificate 1093-20521103}{}{https://www.sololearn.com/Certificate/1093-20521103/pdf/}

\cventry{2021}{Course Machine Learning}{Sololearn platform}{Certificate 1094-20521103}{}{https://www.sololearn.com/Certificate/1094-20521103/pdf/}

\cventry{2021}{Course Python for Beginners}{Sololearn platform}{Certificate 1157-20521103}{}{https://www.sololearn.com/Certificate/1157-20521103/pdf/}

\cventry{2021}{Curso Intermediate Python}{Sololearn platform}{Certificate 1158-20521103}{}{https://www.sololearn.com/Certificate/1158-20521103/pdf/}

\cventry{2021}{Course Python Data Structures}{Sololearn platform}{Certificate 1159-20521103}{}{https://www.sololearn.com/Certificate/1159-20521103/pdf/}

\cventry{2021}{Course Responsive Web Design}{Sololearn platform}{Certificate 1159-20521103}{}{https://www.sololearn.com/Certificate/1162-20521103/pdf/}

\cventry{2021}{Course Kotlin}{Sololearn platform}{Certificate 1160-20521103}{}{https://www.sololearn.com/Certificate/1160-20521103/pdf/}

\cventry{2021}{Course Go}{Sololearn platform}{Certificate 1164-20521103}{}{https://www.sololearn.com/certificates/course/en/20521103/1164/landscape/png}

\cventry{2021}{Course Python for Data Science}{Sololearn platform}{Certificate 1161-20521103}{}{https://www.sololearn.com/certificates/course/en/20521103/1161/landscape/png}

\cventry{2021}{Course Coding for Marketers}{Sololearn platform}{Certificate 1165-20521103}{}{https://www.sololearn.com/Certificate/1165-20521103/pdf/}

\cventry{2021}{Course R}{Sololearn platform}{Certificate 1147-20521103}{}{https://www.sololearn.com/certificates/course/en/20521103/1147/landscape/png}

\cventry{2021}{The distance education model of Cuban higher education in the context of the University of Matanzas}{University of Matanzas}{Matanzas}{Virtual postgraduate course, 1 credit }{http://evead.umcc.cu/course/view.php?id=105}

\cventry{2021}{Course Python for Finance}{Sololearn platform}{Certificate 1139-20521103}{}{https://www.sololearn.com/Certificate/1139-20521103/pdf/}

\cventry{2021}{Course Game Development with JS}{Sololearn platform}{Certificate 1140-20521103}{}{https://www.sololearn.com/Certificate/1140-20521103/pdf/}

\cventry{2022}{Course Web Development Fundamentals course}{Sololearn platform}{Certificate 1141-20521103}{}{https://www.sololearn.com/Certificate/1141-20521103/pdf/}
\section{Thesis}
\cvitem{title}{Mechanism for incorporating SVG graphics into the SCADA GALBA HMI.}
\cvitem{tutors}{Ing. Roberto Cardenas Isla,
	Eng. Ariel Guerra Garayta}
\cvitem{description}{ At the University of Informatics Sciences, Faculty 5 has several production centers
	among which is the Center for Industrial Informatics (CEDIN) where the Guardian of the
	ALBA (GALBA), which had the task of providing a solution to the development of a SCADA in conjunction with the
	PDVSA company.
	The Human Machine Interface or HMI (Human Machine Interface) is the module of the GALBA SCADA system
	that allows configuring, supervising and controlling industrial processes. The HMI provides a library of
	graphic components (CG) that make it possible to recreate as realistically as possible the processes that
	they occur in plants. These graphics can be simple geometric primitives or as complex as
	industrial components used in the industry that you want to monitor. The CGs consist of certain
	limitation due to its current product design and implementation that are defined by specialists and leaves
	open to developers its implementation mechanisms and design.
	The work was based on the development of a mechanism for incorporating SVG graphics into the HMI
	of SCADA GALBA with the purpose of solving the limitations present in the graphic components
	of the HMI. The development of the mechanism is supported by the agile XP methodology with the use of the
	Scalable Vector Graphics (SVG) standard and Qt framework.}

\section{Master's Thesis}
\cvitem{title}{ROFLEXIN/LC System Man-Machine Interface Viewer for Mobile Devices.}
\cvitem{tutors}{Ph.D Ramón Quiza Sardiñas}
\cvitem{description}{  The Center for Studies on Advanced and Sustainable Manufacturing (CEFAS) of the University of Matanzas develops the ROFLEXIN/LC project: Robust, flexible and intelligent, low-cost system for monitoring systems and mechanical processes. This project is fundamentally aimed at small and medium-sized companies, where the use of the alternatives available in the world market is not financially feasible. The present investigation arises from the need of the personnel that uses the ROFLEXIN/LC, to visualize the information collected by said system in mobile devices and tablets. The fundamental objective is to develop a mobile application capable of visualizing the information of the ROFLEXIN/LC system in real time. For the development of the proposed solution, an analysis of the main tools, technologies and methodologies used in the construction of software is carried out. The process was guided by the use of the following tools and technologies: {\em Visual Paradigm} as a CASE tool, UML as a modeling language, JSON as a data exchange format, and Java as a programming language. Developed with the Eclipse IDE with the ADT extension. To validate that the results achieved were those expected, a set of tests was carried out. It was possible to correct all the detected non-conformities, which means that the proposed solution is ready to be deployed. Finally, a Viewer for mobile devices and tablets with Android operating system capable of representing in real time the information coming from the ROFLEXIN/LC system was obtained. }

%\subsection{Vocational}

\section{Experience}
\cventry{2009--2015}{Integral Supervision and Control System of Alba (SCADA GALBA)}{Developer}{Havana}{University of Computer Sciences, Industrial Computer Center}{Module Developer Man-Machine (HMI) belonging to the SCADA Galba project }
\cventry{2014--2016}{Industrial Automated System based on GNU/LINUX (SCADA SAINUX)}{Developer}{Havana}{University of Computer Sciences, Industrial Computer Center}{Module developer Module-Machine (HMI) belonging to the SCADA SAINUX project}
\cventry{2016--2023}{Department of Resources for Learning}{Developer}{Matanzas}{University
	of Matanzas}{Extensions developer to virtual learning environments of the University of Matanzas (Moodle), educational applications for devices with Android operating system. Administrator of virtual learning environments of the University of Matanzas. Installation and deployment of websites with the use of CMS and LMS}
\cventry{2016--2022}{Informachy Department}{Associate Professor}{Matanzas}{University
	of Matanzas}{Teaching category: Instructor}
\cventry{2021--2022}{Combiomed Digital Medical Technology}{Developer}{Havana}{}{Module developer Module-Machine (HMI) of the Software for the Control and Monitoring of the Artificial Fan for COVID-19 patients.}
\cventry{2021--2025}{Pre -University Institute of Exact Sciences of Matanzas}{Competitive programming coach}{Matanzas}{}{}
\cventry{2023--2024}{Network Department}{Specialist B}{Matanzas}{University
	of Matanzas}{Technical support of the Virtual Learning Environments of the University of Matanzas. Installation and deployment of websites using LMS Moodle}
\cventry{2023--2024}{INNOVAT}{Developer}{Matanzas}{}{Design and implementation of the PSGAME+ application for Android mobile devices}
\cventry{2023--2025}{Merchise Autrement}{Developer}{}{\href{https://www.merchise.org/es}{Business Solutions on Odoo, consulting and implementation}}{Implementation of functionalities. Corrections of non-conformities}
\cventry{2024--2025}{Desoft}{Desarrollador}{}{Business Solutions on Odoo}{Implementation of addons.}
\cventry{2025--2025}{Digital Learning}{Desarrollador}{}{\href{https://github.com/DigitalLearning}{Digital Learning}}{Implementación de módulo para \href{https://github.com/DigitalLearning/MDL_tool_quickcoursecreation}{Creación rápida de cursos para plataforma Moodle y Totara}.}

%\subsection{Miscellaneous}
%\cventry{year--year}{Job title}{Employer}{City}{}{General description no longer than 1--2 lines}

\section{Teaching}
\cventry{2016--2016}{Python programming language}{Informatics Engineering}{Matanzas}{University of Matanzas}{5th year}
\cventry{2016--2022}{Computer architecture}{Informatics Engineering}{Matanzas}{University of Matanzas}{2nd year}
\cventry{2017--2020}{Specific preparation for ICPC competitions}{Informatics Engineering}{Matanzas}{University of Matanzas}{1st, 2nd and 3rd year}
\cventry{2019--2019}{Android course}{Bachelor of Computer Education}{Matanzas}{University of Matanzas}{3rd year}
\cventry{2019--2022}{Introduction to programming}{Informatics Engineering}{Matanzas}{University of Matanzas}{1st year}
\cventry{2020--2022}{Design and object -oriented programming}{Informatics Engineering}{Matanzas}{University of Matanzas}{1st year}
\section{Thesis tutoring}
\cventry{2015}{System for the Information Management of Technical Files at the Industrial Computer Center}{Yadriel Cuesta Hechavarría}{Informatic Science University}{Havana}{Diploma work to opt for the title of Computer Science Engineer}
\cventry{2015}{Network State processor for SCADA-GALBA adjustments to the electricity sector}{Carlos Manuel Castillo Chacón, Jose Manuel Acevedo Medina}{Informatic Science University}{Havana}{Diploma work to opt for the title of Computer Science Engineer}


\cventry{2016}{Graphic components for the representation of pipe accessories in the SAINUX SCADA}{Rosmery Pedraza Ceballo}{Informatic Science University}{Havana}{Diploma work to opt for the title of Computer Science Engineer}
\cventry{2016}{Graphic components for the representation of valves and pumps in the SAINUX SCADA}{Claudia María González Fernández}{Informatic Science University}{Havana}{Diploma work to opt for the title of Computer Science Engineer}
\cventry{2016}{Desktop customer for the display environment of the Men's Machine SAINUX SCADA}{Alik Ramón del Risco del Risco}{Informatic Science University}{Havana}{Diploma work to opt for the title of Computer Science Engineer}
\cventry{2016}{Graphic components for the representation of switches and selectors at the SAINUX SCADA}{Javier Bravo Calzado}{Informatic Science University}{Havana}{Diploma work to opt for the title of Computer Science Engineer}
\cventry{2016}{Development of storage graphic components at the SAINUX SCADA}{Luis Orlando Cejas}{Informatic Science University}{Havana}{Diploma work to opt for the title of Computer Science Engineer}
\cventry{2016}{Serialization mechanism for information managed in the editor SAINUX SCADA }{Alvaro Denis Acosta Quesada}{Informatic Science University}{Havana}{Diploma work to opt for the title of Computer Science Engineer}

\cventry{2022}{Plugin to measure the quality indicators of Moodle courses}{Hamsel Brea García}{University of Matanzas}{Matanzas}{Diploma work to opt for the title of Computer Engineering}

\cventry{2022}{Computer System for Human Resources Management and Services of the Artistic Creation Project Vigía}{Ramsey Ricardo Busto Martínez}{University of Matanzas}{Matanzas}{Diploma work to opt for the title of Computer Engineering}

\cventry{2023}{Antencional test system for the “Antonio Guiteras” thermoelectric}{Ronald Guillén Sintes}{University of Matanzas}{Matanzas}{Diploma work to opt for the title of Computer Engineering}
\section{Thesis opponent}
\cventry{2017}{Software for the management of internal client information in hotel systems from the endonarcoeting perspective}{Fabianne Díaz Santiago}{University of Matanzas}{Matanzas}{Diploma work to opt for the title of Computer Engineering}
\cventry{2019}{Computer Information System for the Directorate of the University of Matanzas, SID}{Rasiel Valdespino Jiménez}{University of Matanzas}{Matanzas}{Diploma work to opt for the title of Computer Engineering}

\section{Idioms}
\cvitemwithcomment{Spanish}{Native}{}
\cvitemwithcomment{English}{Professional competence}{}
\section{Tools and technologies}
\cvitemwithcomment{Programming language}{C, C++, Java, JavaScript, Html5, CSS, Python, PHP, SQL, Bash}{}
\cvitemwithcomment{Operating systems}{Debian, Ubuntu, Window, Android}{}
\cvitemwithcomment{Frameworks}{Qt, Yii, CodeIngiter, JQuery, AngularJS, W3CSS, Bootstrap, Django}{}

\cvitemwithcomment{Content management system}{Joomla, Moodle, WordPress}{}

\cvitemwithcomment{Database manager}{Postgres, MySql, MariaDB}{}

\cvitemwithcomment{2D graphic design}{Gimp, Inkscape}{}

\cvitemwithcomment{CASE Tool}{Visual Paradigm}{}

\cvitemwithcomment{Web server}{Apache, XAMPP, NGINX}{}

\cvitemwithcomment{Text processor}{LaTex}{}

\cvitemwithcomment{Version control}{Git}{}
\section{Publications}
\cventry{2017}{Mechanism for the automation of backup copies of the Virtual Learning Environment of the University of Matanzas}{VIII International Scientific Convention of the University of Matanzas }{V International Engineering Workshop}{Matanzas}{Published in citc.umcc.cu with the ISBN code: 978-959-16-3296-8}

\cventry{2017}{"Cesar and the numbers": computer resource to promote the learning of the sum operation, in the first grade}{XIX International Event "Mathematics, Statistics and Computing"}{MATECOMPU 2017}{Varadero}{Published in citc.umcc.cu with the ISBN code: 978-959-16-3671-3}

\cventry{2018}{{\em Support tool for the probability and statistics subject to graph probability distributions in the distance education study modality of the University of Matanzas}}{Opuntia Brava Magazine, Group I Cuban Classification}{}{Las Tunas}{Published in the book Science and technological innovation vol II, in the chapter {\em Distance learning}. With ISBN for the complete work 978-959-7225-26-3, ISBN for the chapter 978-959-7225-34-8}

\cventry{2018}{Development of the Bezier quadratic and cubic curve graphic components}{Monographs}{University of Matanzas}{Matanzas}{Published in monografia.umcc.cu (http://monografias.umcc.cu/monos/2018/FCT/mo1871.pdf)}

\cventry{2018}{Graphical user interface for interaction with SCADA SAINUX modules (VSERSAI)}{Monographs}{University of Matanzas}{Matanzas}{Published in monografia.umcc.cu (http://monografias.umcc.cu/monos/2018/FCT/mo1872.pdf)}

\cventry{2018}{ROFLEXIN/LC system Man-Machine Interface viewer for Android mobile devices}{Monographs}{University of Matanzas}{Matanzas}{Published in monografia.umcc.cu (http://monografias.umcc.cu/monos/2018/FCT/mo1876.pdf)}

\cventry{2019}{Intelligent and low-cost monitoring architecture for mechanical system and processes}{IX International Scientific Conference}{Holguin University}{Holguín}{ISBN:978-959-7237-34-1}

\cventry{2020}{MiGimusic: APK to promote musical education in pioneering Rhythmic Gymnastics }{International Congress online CUBAMTRICIDAD}{University of Sciences of Physical Culture and Sport \emph{Manuel Fajardo}}{Havana}{ISBN:978-959-07-2395-7}

\cventry{2020}{Technological innovation for musical education in Musical Rhythmic Gymnastics}{1st National ONLINE Innovation Workshop}{National Institute of Sports Physical Education and Recreation}{Havana}{ISBN:978-959-203-237-8}

\cventry{2021}{The use of mobile devices in Cuban university teaching}{Academia.edu}{https://www.academia.edu/47857643/EL\_USO\_DE\_DISPOSITIVOS\_MOVILES}{}{}

\cventry{2021}{The use of mobile devices. An innovative experience in professional training}{\emph{Book} The science of the University of Matanzas in the confrontation with COVID-19 }{Editorial UM}{}{}

\section{Events}
\cventry{2008}{Pascal Cup}{University of Computer Sciences}{Havana}{}{Contestant}
\cventry{2009}{Pascal Cup}{University of Computer Sciences}{Havana}{}{Contestant}
\cventry{2010}{Pascal Cup}{University of Computer Sciences}{Havana}{}{Contestant}
\cventry{2011}{Pascal Cup}{University of Computer Sciences}{Havana}{}{Contestant}
\cventry{2011}{Cuban Final of ACM-ICPC}{University of Computer Sciences}{Havana}{}{Contestant}
\cventry{2011}{My website for Cuba}{University of Computer Sciences}{Havana}{}{Relevant}
\cventry{2012}{Pascal Cup}{University of Computer Sciences}{Havana}{}{Contestant}
\cventry{2012}{Student Scientific Conference}{University of Computer Sciences}{Havana}{}{Three relevant, one outstanding and three mentions}
\cventry{2012}{My website for Cuba}{University of Computer Sciences}{Havana}{}{One outstanding and one mention}
\cventry{2013}{Student Scientific Conference}{University of Computer Sciences}{Havana}{}{Outstanding}
\cventry{2015}{Pachamama Game Jam}{University of Computer Sciences}{Havana}{}{Third place}
\cventry{2017}{VI UM Programming Cup}{University of Matanzas}{Matanzas}{}{Third place, Coach}
\cventry{2017}{CEPROMEDE Scientific Conference}{Provincial Center of Sports Medicine}{Matanzas}{}{}
\cventry{2017}{Cuban Final of ACM-ICPC}{Technological University of Havana José Antonio Echeverría}{Havana}{}{Coach}
\cventry{2017}{XIX International Event "The teaching of Mathematics, Statistics and Computing"}{University of Matanzas}{Matanzas}{Paper: {\em César and the numbers}: Computer resource to favor the learning of the operation sum, in the first grade}{}
\cventry{2017}{Cyber-physical Systems Seminar: Modeling and Control}{University of Matanzas}{Matanzas}{}{}
\cventry{2018}{VII UM Programming Cup}{University of Matanzas}{Matanzas}{}{Coach}
\cventry{2018}{XIV Methodological Scientific Conference of the University of Matanzas}{University of Matanzas}{Matanzas}{}{}
\cventry{2018}{Local Constest ACM-ICPC}{University of Matanzas}{Matanzas}{2nd place}{Coach}
\cventry{2018}{ACM-ICPC National Competition}{Military Technical Institute "Jose Martí"}{Havana}{}{Coach}
\cventry{2018}{University Forum of Sciences and Techniques}{University of Matanzas}{Matanzas}{Presentation:{\em Application for monitoring systems and mechanical processes from mobile devices}}{Mention}
\cventry{2018}{II International Symposium of the Network of Researchers of Science and Technology:{\em Science and Technological Innovation}}{}{Las Tunas}{Presentation: {\em Support tool for the probability and statistics subject to graph probability distributions in the distance education study modality of the University of Matanzas}}{}
\cventry{2019}{Seminario {\em Computational intelligence for modeling and control of industrial cyber physical system: Research challenges and opportunities.}}{University of Matanzas}{Matanzas}{}{}
\cventry{2019}{VIII UM Programming Cup}{University of Matanzas}{Matanzas}{}{Coach and organizer}
\cventry{2019}{IX International Scientific Convention}{VI International Engineering Workshop}{University of Matanzas}{Matanzas}{ROFLEXIN/LC system man-machine interface viewer for mobile devices}
\cventry{2019}{IX International Scientific Conference}{}{Holguin University}{Holguín}{Intelligent and low-cost monitoring architecture for mechanical system and processes}
\cventry{2019}{ICPC National Competition}{University of Matanzas}{Matanzas}{}{Coach}

\cventry{2019}{ICPC Caribbean Regional Final}{Eastern University}{Santiago de Cuba}{}{Coach}
\cventry{2020}{First Online Innovation Workshop}{Technological innovation for music education in pioneering rhythmic gymnastics}{INDER}{}{}
\cventry{2021}{ICPC Caribbean Finals (Qualifier)}{Provincial Directorate of Education}{Matanzas}{Preuniversity Teams - Bronze}{Coach}
\cventry{2022}{ICPC Caribbean Finals (Qualifier)}{University of Matanzas}{Matanzas}{}{Coach}
\cventry{2022}{ICPC Caribbean Regional Final}{University of Computer Sciences}{Havana}{}{Coach}
\cventry{2023}{ICPC Caribbean Finals (Qualifier)}{University of Matanzas}{Matanzas}{Preuniversity Teams - Bronze}{Coach}
\cventry{2023}{ICPC Caribbean Regional Final}{University of Computer Sciences}{Havana}{}{Coach}
\section{Research}
\cventry{2020}{Educating with José Martí in the Cuban school}{Education Faculty. University of Matanzas}
{}{}{Collaborator}
\section{Acknowledgments}
\cventry{2013}{Productive results}{}{Havana}{Informatic Science University, Faculty 5}{}
\cventry{2014}{Software developer}{}{Havana}{Informatic Science University, Industrial Computing Center}{}
\cventry{2014}{Outstanding}{}{Havana}{Informatic Science University, Industrial Computing Center}{}
\cventry{2017}{Guest}{ACM-ICPC Caribbean Regional Final}{Villa Clara}{Central University of Las Villas "Marta Abreu"}{}
\cventry{2018}{Guest}{ACM-ICPC Caribbean Regional Final}{Havana}{Informatic Science University}{}
\cventry{2019}{Acknowledgment}{Outstanding Quality Worker}{Matanzas}{University of Matanzas}{}
\cventry{2023}{Acknowledgment}{Competitive programming coach}{Vocational Preuniversity Institute of Exact Sciences \emph{Carlos Marx}}{Matanzas}{}

%\section{Experience}
%\subsection{Vocational}
%\cventry{year--year}{Job title}{Employer}{City}{}{General description no longer than 1--2 lines.\newline{}
%Detailed achievements:
%\begin{itemize}
%\item Achievement 1
%\item Achievement 2 (with sub-achievements)
%  \begin{itemize}
%  \item Sub-achievement (a);
%  \item Sub-achievement (b), with sub-sub-achievements (don't do this!);
%    \begin{itemize}
%    \item Sub-sub-achievement i;
%    \item Sub-sub-achievement ii;
%    \item Sub-sub-achievement iii;
%    \end{itemize}
%  \item Sub-achievement (c);
%  \end{itemize}
%\item Achievement 3
%\item Achievement 4
%\end{itemize}}
%\cventry{year--year}{Job title}{Employer}{City}{}{Description line 1\newline{}Description line 2\newline{}Description line 3}
%\subsection{Miscellaneous}
%\cventry{year--year}{Job title}{Employer}{City}{}{Description}

%\section{Languages}
%\cvitemwithcomment{Language 1}{Skill level}{Comment}
%\cvitemwithcomment{Language 2}{Skill level}{Comment}
%\cvitemwithcomment{Language 3}{Skill level}{Comment}
%\cvitemwithcomment{Language 4}{Skill level}{Comment}

%\section{Computer skills}
%\cvdoubleitem{category 1}{XXX, YYY, ZZZ}{category 4}{XXX, YYY, ZZZ}
%\cvdoubleitem{category 2}{XXX, YYY, ZZZ}{category 5}{XXX, YYY, ZZZ}
%\cvdoubleitem{category 3}{XXX, YYY, ZZZ}{category 6}{XXX, YYY, ZZZ}

%\section{Skill matrix}
%\cvitem{Skill matrix}{Alternatively, provide a skill matrix to show off your skills}
%% Skill matrix as an alternative to rate one's skills, computer or other. 

%% Adjusts width of skill matrix columns. 
%% Usage \setcvskillcolumns[<width>][<factor>][<exp_width>]
%% <width>, <exp_width> should be lengths smaller than \textwidth, <factor> needs to be between 0 and 1.
%% Examples:
% \setcvskillcolumns[5em][][]%    adjust first column. Same as \setcvskillcolumns[5em]
% \setcvskillcolumns[][0.45][]%   adjust third (skill) column. Same as \setcvskillcolumns[][0.45]
% \setcvskillcolumns[][][\widthof{``Year''}]%     adjust fourth (years) column.
% \setcvskillcolumns[][0.45][\widthof{``Year''}]%
% \setcvskillcolumns[\widthof{``Languag''}][0.48][]
% \setcvskillcolumns[\widthof{``Languag''}]%

%% Adjusts width of legend columns. Usage \setcvskilllegendcolumns[<width>][<factor>]
%% <factor> needs to be between 0 and 1. <width> should be a length smaller than \textwidth
%% Examples:
% \setcvskilllegendcolumns[][0.45]
% \setcvskilllegendcolumns[\widthof{``Legend''}][0.45]
% \setcvskilllegendcolumns[0ex][0.46]% this is usefull for the banking style

%% Add a legend if you are using \cvskill{<1-5>} command or \cvskillentry
%% Usage \cvskilllegend[*][<post_padding>][<first_level>][<second_level>][<third_level>][<fourth_level>][<fifth_level>]{<name>}
% \cvskilllegend % insert default legend without lines
%\cvskilllegend*[1em]{}% adjust post spacing
% \cvskilllegend*{Legend}%  Alternatively add a description string
%% adjust the legend entries for other languages, here German
% \cvskilllegend[0.2em][Grundkenntnisse][Grundkenntnisse und eigene Erfahrung in Projekten][Umfangreiche Erfahrung in Projekten][Vertiefte Expertenkenntnisse][Experte\,/\,Spezialist]{Legende}

%% Alternative legend style with the first three skill levels in one column
%% Usage \cvskillplainlegend[*][<post_padding>][<first_level>][<second_level>][<third_level>][<fourth_level>][<fifth_level>]{<name>}
% \setcvskilllegendcolumns[][0.6]%  works for classic, casual, banking
% \setcvskilllegendcolumns[][0.55]%  works better for oldstyle and fancy
% \cvskillplainlegend{}
% \cvskillplainlegend[0.2em][Grundkenntnisse][Grundkenntnisse und eigene Erfahrung in Projekten][Umfangreiche Erfahrung in Projekten][Vertiefte Expertenkenntnisse][Experte/Guru]{Legende}

%% Add a head of the skill matrix table with descriptions.
%% Usage \cvskillhead[<post_padding>][<Level>][<Skill>][<Years>][<Comment>]%
%\cvskillhead[-0.1em]%   this inserts the standard legend in english and adjust padding
%% Adjust head of the skill matrix for other languages
% \cvskillhead[0.25em][Level][F\"ahigkeit][Jahre][Bemerkung]

%% \cvskillentry[*][<post_padding>]{<skill_cathegory>}{<0-5>}{<skill_name>}{<years_of_experience>}{<comment>}% 
%% Example usages:
%\cvskillentry*{Language:}{3}{Python}{2}{I'm so experienced in Python and have realised a million projects. At least.}
%\cvskillentry{}{2}{Lilypond}{14}{So much sheet music! Man, I'm the best!}
%\cvskillentry{}{3}{\LaTeX}{14}{Clearly I rock at \LaTeX}
%\cvskillentry*{OS:}{3}{Linux}{2}{I only use Archlinux btw}% notice the use of the starred command and the optional 
%\cvskillentry*[1em]{Methods}{4}{SCRUM}{8}{SCRUM master for 5 years}
%% \cvskill{<0-5>} command
% \cvitem{\textbackslash{cvskill}:}{Skills can be visually expressed by the \textbackslash{cvskill} command, e.g. \cvskill{2}}

%\section{Interests}
%\cvitem{hobby 1}{Description}
%\cvitem{hobby 2}{Description}
%\cvitem{hobby 3}{Description}

%\section{Extra 1}
%\cvlistitem{Item 1}
%\cvlistitem{Item 2}
%\cvlistitem{Item 3. This item is particularly long and therefore normally spans over several lines. Did you notice the indentation when the line wraps?}

%\section{Extra 2}
%\cvlistdoubleitem{Item 1}{Item 4}
%\cvlistdoubleitem{Item 2}{Item 5\cite{book2}}
%\cvlistdoubleitem{Item 3}{Item 6. Like item 3 in the single column list before, this item is particularly long to wrap over several lines.}

%\section{References}
%\begin{cvcolumns}
%  \cvcolumn{Category 1}{\begin{itemize}\item Person 1\item Person 2\item Person 3\end{itemize}}
%  \cvcolumn{Category 2}{Amongst others:\begin{itemize}\item Person 1, and\item Person 2\end{itemize}(more upon request)}
%  \cvcolumn[0.5]{All the rest \& some more}{\textit{That} person, and \textbf{those} also (all available upon request).}
%\end{cvcolumns}

% Publications from a BibTeX file without multibib
%  for numerical labels: \renewcommand{\bibliographyitemlabel}{\@biblabel{\arabic{enumiv}}}% CONSIDER MERGING WITH PREAMBLE PART
%  to redefine the heading string ("Publications"): \renewcommand{\refname}{Articles}
%\nocite{*}
%\bibliographystyle{plain}
%\bibliography{publications}                        % 'publications' is the name of a BibTeX file

% Publications from a BibTeX file using the multibib package
%\section{Publications}
%\nocitebook{book1,book2}
%\bibliographystylebook{plain}
%\bibliographybook{publications}                   % 'publications' is the name of a BibTeX file
%\nocitemisc{misc1,misc2,misc3}
%\bibliographystylemisc{plain}
%\bibliographymisc{publications}                   % 'publications' is the name of a BibTeX file

%\clearpage
%-----       letter       ---------------------------------------------------------
% recipient data
%\recipient{Company Recruitment team}{Company, Inc.\\123 somestreet\\some city}
%\date{January 01, 1984}
%\opening{Dear Sir or Madam,}
%\closing{Yours faithfully,}
%\enclosure[Attached]{curriculum vit\ae{}}          % use an optional argument to use a string other than "Enclosure", or redefine \enclname
%\makelettertitle

%Lorem ipsum dolor sit amet, consectetur adipiscing elit. Duis ullamcorper neque sit amet lectus facilisis sed luctus nisl iaculis. Vivamus at neque arcu, sed tempor quam. Curabitur pharetra tincidunt tincidunt. Morbi volutpat feugiat mauris, quis tempor neque vehicula volutpat. Duis tristique justo vel massa fermentum accumsan. Mauris ante elit, feugiat vestibulum tempor eget, eleifend ac ipsum. Donec scelerisque lobortis ipsum eu vestibulum. Pellentesque vel massa at felis accumsan rhoncus.

%Suspendisse commodo, massa eu congue tincidunt, elit mauris pellentesque orci, cursus tempor odio nisl euismod augue. Aliquam adipiscing nibh ut odio sodales et pulvinar tortor laoreet. Mauris a accumsan ligula. Class aptent taciti sociosqu ad litora torquent per conubia nostra, per inceptos himenaeos. Suspendisse vulputate sem vehicula ipsum varius nec tempus dui dapibus. Phasellus et est urna, ut auctor erat. Sed tincidunt odio id odio aliquam mattis. Donec sapien nulla, feugiat eget adipiscing sit amet, lacinia ut dolor. Phasellus tincidunt, leo a fringilla consectetur, felis diam aliquam urna, vitae aliquet lectus orci nec velit. Vivamus dapibus varius blandit.

%Duis sit amet magna ante, at sodales diam. Aenean consectetur porta risus et sagittis. Ut interdum, enim varius pellentesque tincidunt, magna libero sodales tortor, ut fermentum nunc metus a ante. Vivamus odio leo, tincidunt eu luctus ut, sollicitudin sit amet metus. Nunc sed orci lectus. Ut sodales magna sed velit volutpat sit amet pulvinar diam venenatis.

%Albert Einstein discovered that $e=mc^2$ in 1905.

%\[ e=\lim_{n \to \infty} \left(1+\frac{1}{n}\right)^n \]

%\makeletterclosing

%\clearpage\end{CJK*}                              % if you are typesetting your resume in Chinese using CJK; the \clearpage is required for fancyhdr to work correctly with CJK, though it kills the page numbering by making \lastpage undefined
\end{document}


%% end of file `template.tex'.

